\documentclass[11pt]{article}

\usepackage[top=2.5cm,bottom=2.5cm,left=2cm,right=2cm]{geometry}
\usepackage[utf8]{inputenc}
\usepackage{enumitem}
\usepackage{parskip}
\usepackage{booktabs}
\usepackage{xcolor}
\usepackage{hyperref}
\usepackage[acronym]{glossaries}
\usepackage{pgfgantt}
\usepackage{pdflscape}
\usepackage{amsmath}
\usepackage{tikz}
\usepackage{rotating}
\usepackage{array}
\usepackage{graphicx}
\usepackage{caption}
\usepackage{subcaption}

\setlist[itemize]{noitemsep,nolistsep}
\setlist[enumerate]{noitemsep,nolistsep}

\hypersetup{
  colorlinks,
  linkcolor={black!40!black},
  urlcolor={blue!40!black},
  citecolor={blue!40!black}
}

\makenoidxglossaries
\newacronym{SPICE}{SPICE}{Simulation Program with Integrated Circuit Emphasis}
\newacronym{CAD}{CAD}{Computer-aided Design}
\newacronym{SDK}{SDK}{Software Development Kit}
\newacronym{AR}{AR}{augmented reality}
\newacronym{VR}{VR}{virtual reality}
\newacronym{GUI}{GUI}{graphical user interface}
\newacronym{NN}{NN}{neural network}
\newacronym{CNN}{CNN}{convolutional neural network}
\newacronym{HDL}{HDL}{hardware description language}

\title{Designing a Sketch-Based Interface for Electronic Circuit Simulation}
\author{Taharka Okai}
\date{April 2023}

\begin{document}

\begin{titlepage}
    \begin{center}
        \vfill
        \includegraphics[width={0.7\textwidth}]{../graphics/university-of-bristol-logo-png-transparent}

        \vfill
        {\Huge{Designing a Sketch-based Interface for Electronic Circuit Simulation}}

        \vfill
        {\Large\textbf{Taharka Okai}}

        \vspace{1cm}
        {\Large{\textbf{April 2023}}}

        \vfill

        \vspace{1cm}
        {\Large{Final year project thesis submitted in support of the
                degree of Master of Engineering in Computer Science and Electronics}}

        \vspace{1cm}
        {\Large{Department of Electrical \& Electronic Engineering}}

        {\Large{University of Bristol}}
    \end{center}
\end{titlepage}

% \begin{figure*}[hptb]
%     \centering
%     \includegraphics[width={0.7\textwidth}]{../graphics/university-of-bristol-logo-png-transparent}
% \end{figure*}

\thispagestyle{empty}
\setcounter{page}{1}

\pagebreak
\tableofcontents
\printnoidxglossary[type=\acronymtype,title=List of Acronyms]

\pagebreak
\begin{abstract}

\end{abstract}

\pagebreak
\section{Introduction}
\label{Introduction}

\pagebreak
\section{Related Work and Key References}
\label{subsec:Related Work and Key References}

\subsection{Data Acquisition, Input and Processing}
\label{subsec:Data Acquisition, Input and Processing}

The paper by Bonnici et al. \cite{101017S} is a review paper that defines the landscape of sketch-based
interfaces, with an emphasis on mechanical systems. They list the challenges faced with such approaches,
and solutions taken by various authors. In particular, they provide a helpful categorisation of input acquisition
and processing techniques. They define `Paper Sketches' and `Digital Sketches' which are input techniques that
use digital images captured of a non-electronic sketch (i.e., with pen or pencil on paper) and otherwise
direct digital input by the use of a peripheral device (i.e., with a digital pen or touch screen).

With regard to processing, Bonnici et al. describe the pipeline of image processing for each input approach, which
broadly falls into the following order: Binarisation, Vectorisation and Interpretation. These refer to distinguishing
foreground from background elements, smoothing line strokes and creating a physical system respectively.

In the paper by Costa et al. \cite{109781I}, they present SketchyDynamics, an application that takes in
user input sketches via a touch screen, builds a system from the sketch and allows simulation of that system.
They did not need to acquire any data to interpret user input as they use a gesture recogniser known as CALI
to turn user sketches into shapes, interpreted as simulation primitives. They also did not need to have significant
development time dedicated to processing the system as they used the Box2D physics engine, which is a rigid
body simulation library for games.

Hu et al., \cite{6274802} use a game engine, Unity3D, and the accompanying physics tools it provides to create a
mechanical system simulation. As such, they do not provide a means of data acquisition and input, as the systems
and the entities within are 3D modelled in a separate software, 3dmax. However, this reveals the issue of decoupling in
mechanical physical simulations -- the simulator and 3D modelling steps are separate and often require separate software.

Bergig et al. \cite{5336490} present a software project that can `analyse, visualise and simulate mechanical system in 3D'.
They capture input from a webcam and display it to the user. This is an example of a `Paper Sketch' implementation, as the
sketch system is prepared on a non-digital medium. They make use of orthographic projection techniques to produce simple systems
of the sketches. Rotations, forces, friction and other physical properties are inferred from annotations on the diagram, a feature
unique to this paper.

Pichiliani et al. \cite{5460522} recognise the ease of sketching in the design phase of physical concept and discuss
the use of `Digital Ink' applications, which is an example of the `Digital Sketches' concept mentioned in the review in
\cite{101017S}. They also propose collaborative input, where multiple engineers can contribute to a schematic at the same
time. While this is a useful feature, it is not within scope of the current project and could be considered for future
work. Their input method uses the Microsoft Tablet \gls{SDK} which was shared under a licence agreement. With this, they
are able to construct basic shapes for which to do 2D mechanical simulations.

Fang et al., \cite{4722231} present Sketch3D, which also uses a digital interface in order to retrieve user sketches.
They process the data by using a curve estimation algorithm which involves preprocessing the strokes on the screen, finding
key points, recognising primitives (shapes), and reconstruction.

\subsection{Electronic Systems}
\label{subsec:Electronic Systems}

There are a variety of robust electronic simulation tools such as Micro-Cap, AWR Design Environment, \gls{SPICE}-based
tools such as LTSpice and more \cite{web:micro-cap,web:awrde,web:ltspice}. However, each of the input modes provided by
these applications do not include sketch-based user input. Despite that, their powerful simulation capabilities make them
useful to apply to this project, in terms of the algorithms required to simulate a given system and produce time-based output.

For a more targeted case study, SIMULINK uses a drag-and-drop type interface to build block diagrams of electronic systems
\cite{web:simulink}. This approach removes any need to code and is similar in the approaches
taken by the aforementioned electronic system simulation tools. It also provides the ability to link the input and output of
models together for use in more complex systems, which allows for a modular design approach.

Certain \gls{SPICE}-based tools are able to import and export a circuit description file, or netlist \cite{web:cadence-pspice}. It may be possible,
then, to offload simulation tasks to pre-existing software. This turns the problem of sketch-based simulation into generating a
valid circuit description that can be used by an external program. This technique is apparent in the web-based
tool developed by Kadlec et al. \cite{5432817}, where they allow users to edit subcircuit parameters directly in the model description.
Also in this paper, the authors describe the architecture of their software tool, where, instead of a sketch-based interface, there is a
graphical user interface accessible through a web browser.

To avoid the introduction of any dependency on external software, this can instead be implemented directly, with the algorithms used
by \gls{SPICE}-based tools, which involves a combination of nodal analysis and numerical methods to solve the system of equations defined
by the netlist \cite{web:spice-algorithm,book:circuit-simulation}.

\subsection{Concept-Adjacent Success}
\label{subsec:Concept-Adjacent Success}

In markedly similar projects to this one proposed, such the one presented by Zamora et al. \cite{20090001}, there is a successful implementation
of a sketch-based simulation tool for logic circuits. They make use of digital sketching as their interface and employ a recognition
pipeline in the model generation process. They describe the segmentation of input strokes, then the interpretation of these strokes
into primitives, which for this project are a set of logic gates. Once the primitives have been defined in the sketch using a classifier,
then the circuit is evaluated which is an underlying node graph representation. This is one such example that makes use of the techniques
described in the prior sections and represents an outcome similar to the aims of this project.

Similarly, with the project by Alvarado et al. \cite{9155937}, they also include error correction to primitives which attempt to improve
the input sketch, by notifying the user that endpoints are not properly connected, for example.

The project presented by Dreijer \cite{dreijer}, they produce an idealised version of the type of tool that this project
is aiming to produce, using digital ink as their medium. Their implementation consists of a drawing process that accepts strokes
until a symbol has been created before creating a new symbol, a clustering process that identifies symbols by finding certain types
of line intersection types within a specific region, and a design for a format of a circuit description, similar to \gls{SPICE}-based
netlist files.

Finally, the project described by Majeed et al. \cite{191016} takes the ideas by Dreijer and compare a variety of machine learning and deep
learning tactics, producing Sketic. Notably, they highlight the fact that deep learning approaches implicitly extract features from the user's
sketches, decreasing the dependency on image preprocessing. When compared with a \gls{CNN} classifier over a regular \gls{NN} classifier, they
tout improved accuracy and speed, at the cost of additional overhead that scales with the complexity of the input diagram. Additionally,
their software tool has the ability to produce \gls{HDL} code (in the form of Verilog \cite{8299595}) for the logic circuits defined in the sketch, which is a direct analogue to
producing a netlist for an electronics circuit developed using a \gls{SPICE} tool. This paper is the most recent
one researched with high relevancy (2020), making use of modern techniques and advancements in the field that some older papers mentioned have not demonstrated.

\subsection{Research Summary}
\label{subsec:Research Summary}

Table~\ref{tab:Table of research summaries} contains relevant papers discussed prior, and their estimated utility as a foundation for this project.
This table uses a Relevance vs. Risk evaluation to rank each paper's contribution to the background knowledge behind this research
topic. Relevance describes the amount of directly applicable content the paper has, and Risk is a factor describing complexity
(high = low risk).

The preliminary research conducted in the above section has yielded the following useful information:

\begin{itemize}
    \item Considerations required for different input media types,
    \item Separate challenges faced simulation electronic and mechanical systems,
    \item Challenges faced when processing sketches generating primitives,
    \item Processes of generating a model from a sketch using various classification algorithms,
    \item Methods of simulating a model using numerical methods,
    \item Potential to offload simulation tasks to an external program,
    \item A set of concept-adjacent and concept-aligned software tools as a potential basis.
\end{itemize}

\begin{table}[htpb]
    \footnotesize\centering
    \begin{tabular}{
        @{}
        l
        >{\raggedright\arraybackslash}p{7cm}
        l
        @{}
        }
        Paper                                                  &
        Summary                                                  \\
        \midrule
        Where do we stand? Bonnici et al. \cite{101017S}       &
        Review: processing sketches, input/model classification  \\
        \\

        SketchyDynamics, Costa et al. \cite{109781I}           &
        A novel sketch-based approach, mechanics simulator       \\
        \\

        Unity3D approach, Hu et al. \cite{6274802}             &
        Using Unity3D game engine for simulation, no sketching   \\
        \\

        \Gls{AR}/\gls{VR}, Bergig et al. \cite{5336490}        &
        Mechanics simulator with \gls{AR}/\gls{VR} capabilities  \\
        \\

        Collaborative Design, Pichiliani et al. \cite{5460522} &
        Collaborative sketch-based mechanics simulator           \\
        \\

        Sketch3D, Fang et al. \cite{4722231}                   &
        Mechanics simulator, sketch processing techniques        \\
        \\

        Internet-based Sketch Tool, Hu et al. \cite{5432817}   &
        Electronic system simulator hosted online                \\
        \\

        CircuitBoard, Zamora et al. \cite{20090001}            &
        Sketch-based logic simulator                             \\
        \\

        LogiSketch, Alvarado et al. \cite{9155937}             &
        Sketch-based logic simulator                             \\
        \\

        Electronics model generator, Dreijer \cite{dreijer}    &
        Sketch-based electronics model generator with detailed
        input processing techniques                              \\
        \\

        Sketic, Majeed et al. \cite{191016}                    &
        Sketch-based electronics model simulator making use of
        \gls{CNN}s and VHDL code generation                      \\
        \\
    \end{tabular}
    \caption[Table of research summaries]{
        Table of research summaries.}
    \label{tab:Table of research summaries}
\end{table}


\pagebreak
\section{Method}
\label{sec:Method}

\subsection{Resource Requirements}
\label{subsec:Resource Requirements}

The following table (Table \ref{tab:Table of project resources}) summarises the expected resource list for this
project.

\begin{table}[hptb]
    \footnotesize\centering
    \begin{tabular}{
        @{}
        >{\raggedright\arraybackslash}p{5cm}
        >{\raggedright\arraybackslash}p{2cm}
        >{\raggedright\arraybackslash}p{5cm}
        >{\raggedright\arraybackslash}p{4cm}
        @{}}
        Resource                               & Type     & Description                                        & Availability                          \\
        \toprule
        Workstation                            & Hardware & For computation and compilation                    & Own or Laboratory                     \\
        Smartphone or Tablet (with camera)     & Hardware & For development, testing and demonstration         & Own or Laboratory                     \\
                                               &          &                                                    &                                       \\
        Machine/Deep Learning Tools            & Software & Development library, e.g. PyTorch                  & Online                                \\
        Image Processing Tools                 & Software & Development library, e.g. OpenCV                   & Online                                \\
        Software Development Kit and framework & Software & Cross-platform development, e.g. Flutter           & Online                                \\
        Mobile Device emulator                 & Software & E.g. Android Studio for cross-platform development & Online                                \\
                                               &          &                                                    &                                       \\
        Labelled data set                      & Misc     & For training the sketch-to-system model            & Own or Office Supplies                \\
        Circuit Simulation Guide               & Misc     & Insight into the algorithms used in simulation     & Bristol University Library (Acquired) \\
    \end{tabular}
    \caption{Table of project resources}
    \label{tab:Table of project resources}
\end{table}

\pagebreak
\section{Results}
\label{sec:Results}

\pagebreak
\section{Discussion}
\label{sec:Discussion}

\pagebreak
\section{Conclusion}
\label{sec:Conclusion}

\pagebreak
\bibliographystyle{IEEEtran}
{\raggedright \bibliography{project}}

\end{document}